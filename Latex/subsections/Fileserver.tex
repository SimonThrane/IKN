I serveren blev, der implementeret en konstruktor og sendFile-metode.

Konstruktoren er lavet, så den opretter en ny TCPListener, der lytter på en vilkårlig IP-adresse på en hard-coded Port.
Derefter opretter den en . Serveren startes og venter på, at den opretter en TCP-connection med en client.
Når forbindelsen med en client er oprettet den en networkstream til clientens socket. Serveren læser derefter, hvilken fil clienten ønsker overført. 
Derefter tjekker serveren om den ønskede fil findes og sender dette resultat til clienten.
Hvis filen findes sendes denne til clienten ved at kalde metoden sendFile.

sendFile (String fileName, long fileSize, NetworkStream io) opretter et array af bytes med en faststørrelse for at overføre filen i stykker af 1000 bytes såfremt dette er muligt.
Filen bliver læst ind ved brug af FileStream kommandoer og lagt i en variable, derefter overføres filen over networkStreamen i mindre stykker.