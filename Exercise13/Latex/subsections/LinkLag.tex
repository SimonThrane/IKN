Link laget er i den udviklede protocol stak det nederste lag, og ligger lige over stakkens fysiske lag. I det udleverede matriale var en smule af link laget allerede lavet, så det af gruppen implementerede er hhv. send og recieve metoden.

\subsection{Send metoden}

\begin{verbatim}
public void send (byte[] buf, int size)
\end{verbatim}

I send metoden er det væsentlige at køres en SLIP metode på beskeden der skal sendes. Dette vil sige at beskeden skal omkrænses af to delimiters, som i dette tilfælde er karakteren 'A'. Da protokollen stadig skal være i stand til at sende 'A' inde i beskeden, skal et hvert 'A' udskiftes med et "BC" og et 'B' skal deror udskiftes til et "BD". Denne algoritme kaldes en SLIP metode.


\begin{lstlisting}
	int j=1;
	for (int i = 0; i < size; i++) {
		if(buf[i] == (byte)('A'))
			{
			buffer[j++]=Convert.ToByte('B');
			buffer[j++]=Convert.ToByte('C');

			}
		else if(buf[i] == (byte)('B'))
			{
			buffer[j++]=Convert.ToByte('B');
			buffer[j++]=Convert.ToByte('D');
			}
		else{
			buffer [j++] = buf [i];
			}
		}
\end{lstlisting}

Som det kan ses i koden ovenfor holdes der styr på både den af transportlaget givne streng, og den streng der skal sendes videre til det fysiske lag. Der anvendes en allerede lavet buffer der er garanteret at være stor nok (kan ikke ses i kodenlistingen).\\

\noindent Læg mærke til at j starter ved 1 for at der således senere kan sættes 'A' ind på den forreste plads. Koden for sendingen og delimiterindsættelse ses nedenfor.

\begin{verbatim}
buffer[0]=Convert.ToByte('A');
buffer[j++]=Convert.ToByte('A');
serialPort.Write (buffer, 0, j);
\end{verbatim}

\subsection{Receive metoden}
Nu indspiceres den modtagende del af protokolstakkens link lag. Link laget bruger de tidligere omtalte delimiters til at afgrænse den komne besked, og andvender en antiSLIP metode, som operere modsat af SLIP metoden. Nedenfor ses hele recieve metoden, hvor det kan ses at der ventes på det første 'A', og derefter indsættes den modtagne besked i en lokal buffer. Hele beskeden er modtaget når det næste 'A' læses, og der køres antiSLIP metoden, som også sætter den modtagne frame ind i den af parametre givne buffer.  Variablen f bliver brugt til at holde styr på hvor langt man er nået i indsættelsen af bufferen, og dermed også størrelsen af beskeden der modtages. Størrelsen på denne buffer returneres så transportlaget kender størrelsen af den modtagne frame.

\begin{verbatim}
public int receive (ref byte[] buf)
{
while ((char) serialPort.ReadByte () != 'A') {}
int i = 0;
while ((buffer[i++]= (byte) serialPort.ReadByte () )!= (byte)'A') {}

i--;
int f = 0;

for(int j =0; j<i; j++)
{
if(buffer[j]==(byte)('B'))
{
j++;
if (buffer [j]==(byte)('C'))
buf [f++] = Convert.ToByte ('A');
if(buffer[j]==(byte)('D'))
buf [f++] = Convert.ToByte ('B');
}
else
{
buf [f++] = buffer [j];
}
}

return f;

}
\end{verbatim}

Transportlaget kalder denne recieve metode, og giver selv den buffer med, den ønsker at modtage på.