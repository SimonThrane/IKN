\chapter{Konklusion}\label{ch:konclusion}
I denne opgave er en UDP server og client blevet implementeret. Implementationen er lavet i C{\#}. Clienten er lavet, så den sender to argumenter til UDP serveren. Første argument er IP-adressen på den server den prøver at koble til, og det andet argument er givet med for at kunne fortælle hvad man ønsker af serveren, men som basalt set bare er en tekststreng. Serveren svarer så på clienten afhængigt af om den modtager et 'l' eller et 'u', sender serveren data fra "/proc/loadavg" eller "/proc/uptime". Alt dette sker på en hardcoded port (9000). Desuden er serveren sat til at lytte bagefter, så den virker iterativt.
Den udarbejde server og client blev testet ved at sende forskellige kommandoer i Linux for at teste, at de hentede data passede med den forespørgsel der blev sendt fra clienten og desuden at serveren svarede hvis den fik en forkert kommando. Under arbejdet er der opnået erfaringer med: UDP og Socket Programmering.