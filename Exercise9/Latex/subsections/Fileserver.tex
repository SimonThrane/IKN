Der er blevet implementeret en UDP server. Denne server lytter på en fast port (9000) og generelt bruger bruges bibliotek funktionerne fra System.Net.Sockets meget i denne implementering. Først og fremmest startes serveren og en udp client oprettes så der kan lyttes på port 9000. Der er her ikke tale om en reel klient men blot en UDP instans.

\begin{verbatim}
Console.WriteLine(" >> Server Started");
UdpClient udpClient = new UdpClient(PORT);
\end{verbatim}

\noindent Nu er serveren initializeret så det der sker herefter ligger alt sammen i en uendelig løkke. Derefter sættes der et endpoint som gør at der kan recieves fra det den client der bliver fundet. Endpointet lytter på port 9000 efter enhver IP-adresse. Derefter bliver det modtagne data lavet til en tekst streng, istedet for et bytearray.  

\begin{verbatim}
IPEndPoint RemoteIpEndPoint = new IPEndPoint(IPAddress.Any,PORT);
Byte[] receiveBytes = udpClient.Receive(ref RemoteIpEndPoint);
string recievedString = Encoding.ASCII.GetString(receiveBytes);
\end{verbatim}

\noindent Derefter sættes instansen af UDP til at connecte til det endpoint der er blevet modtaget fra, og afhængigt af om der modtages 'l' og 'u', sendes der enten data fra serverens /proc/loadavg eller /proc/uptime. Der er brugt ToLower() metoden, så der ikke er case sensitivity.

\begin{verbatim}
udpClient.Connect (RemoteIpEndPoint);
switch (recievedString.ToLower()) {
	case "l":
	udpClient.Send(ReadFile("/proc/loadavg"), ReadFile("/proc/loadavg").Length);
	break;
	case "u":
	udpClient.Send(ReadFile("/proc/uptime"), ReadFile("/proc/uptime").Length);
	break;
	default:
	Byte[] sendBytes = Encoding.ASCII.GetBytes("File do not exist!");
	udpClient.Send(sendBytes, sendBytes.Length);
	break;
\end{verbatim}

\noindent Der er blevet implementeret en hjælpemetode ReadFile() som bruges til at hente dataen fra de nævnte filer.

\begin{verbatim}
private byte[] ReadFile(string path)
{
using(var streamReader = new StreamReader (path))
{
string fileContent = streamReader.ReadToEnd ();
Byte[] buffer = Encoding.ASCII.GetBytes(fileContent);
return buffer;
}
}
\end{verbatim}

\noindent Som tidligere nævnt virker serveren iterativt fordi den lytter på enhver port og begynder at lytte igen efter hver client interaktion. Desuden lukkes serveren efter den uendelige løkke, af almindelig god skik.


%I serveren blev, der implementeret en konstruktor og sendFile-metode. Samt indsat en del API funktioner. 
%Konstruktoren er lavet, så den opretter en ny TCPListener, der lytter på en vilkårlig IP-adresse på en hard-coded Port.
%Nu sættes serveren til at oprette en TCP-connection med en client.
%\begin{verbatim}
%while (true) {
%try
%{
%clientSocket = serverSocket.AcceptTcpClient ();
%NetworkStream networkStream = clientSocket.GetStream();
%string dataFromClient = LIB.readTextTCP(networkStream);
%Console.WriteLine(" >> Data from client - " + dataFromClient);
%FileInfo fileInfo = new FileInfo(dataFromClient);
%long fileSize = fileInfo.Length;
%LIB.writeTextTCP(networkStream,fileSize.ToString());
%if(0!=LIB.check_File_Exists(dataFromClient))
%{
%sendFile(dataFromClient,fileSize,networkStream);
%}
%networkStream.Flush();
%}
%\end{verbatim}
%
%På koden ovenfor kan ses at når forbindelsen med en client oprettes der en networkstream til clientens socket. Serveren læser derefter, hvilken fil clienten ønsker overført. 
%Derefter tjekker serveren om den ønskede fil findes og sender dette resultat til clienten.
%Hvis filen findes sendes denne til clienten ved at kalde metoden sendFile.
%
%sendFile (String fileName, long fileSize, NetworkStream io) opretter et array af bytes med en fast størrelse for at overføre filen i stykker af maksimalt 1000 bytes såfremt dette er muligt. Kode segmentet kan ses nedenfor.
%
%\begin{verbatim}
%private void sendFile (String fileName, long fileSize, NetworkStream io)
%{
%Byte[] buffer = new Byte[BUFSIZE];
%
%var fileStream = new FileStream (fileName,FileMode.Open,FileAccess.Read);
%
%
%for (int len = fileStream.Read(buffer,0,BUFSIZE); len != 0; len =
%fileStream.Read(buffer,0,BUFSIZE))
%{
%io.Write (buffer, 0, len);
%}
%
%}
%\end{verbatim}
%Filen bliver læst ind ved brug af FileStream kommandoer og lagt i en variable, derefter overføres filen over networkStreamen i mindre stykker.
%
%I selve io.Write() metoden kan det ses at 'len' er antallet af bytes der rent faktisk er indlæst af filestreamen, så selvom det forsøges at læse 1000 bytes, vil metoden kun skrive det antal bytes der faktisk er indlæst i bufferen, og derfor skrives ikke 1000, men maksimalt 1000 bytes.
%
%
%Når hele overførslen fra serveren er overstået, sættes serveren til igen at lytte efter nye TCP forbindelser. Dette kan ses i det første kodesegment i afsnittet. På denne måde er serveren gjort iterativ.