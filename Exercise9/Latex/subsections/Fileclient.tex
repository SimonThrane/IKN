
\noindent Der er blevet implementer en UDP client som der kan sende ved brug af to parametre kan sende en streng til en vilkårlig IP-adresse, på en fastlagt port (9000). Derefter lytter clienten efter svar fra serveren.\\
\noindent Først oprettes en instans af System.Net.Sockets udpclient, med port 9000 som argument. Derefter forsøges det at connecte til det første brugermedgivne argument i programmet (IP-adressen), og hvis det lykkedes sendes det andet brugermedgivne argument til den givne IP adresse, som et bytearray.  
\begin{verbatim}
UdpClient udpClient = new UdpClient(PORT);
try{
udpClient.Connect(args[0], PORT);
// Sends a message to the host to which you have connected.
Byte[] sendBytes = Encoding.ASCII.GetBytes(args[1]);
udpClient.Send(sendBytes, sendBytes.Length);
\end{verbatim}

Nu sættes clienten til at lytte. Den lytter fra alle IP-addresser, men kun på port 9000. Når clienten modtager noget laver den det om til en streng og udskriver det i konsolen. Derefter lukkes forbindelsen i udpclienten.

\begin{verbatim}
//IPEndPoint object will allow us to read datagrams sent from any source.
IPEndPoint RemoteIpEndPoint = new IPEndPoint(IPAddress.Any, PORT);

// Blocks until a message returns on this socket from a remote host.
Byte[] receiveBytes = udpClient.Receive(ref RemoteIpEndPoint); 
string returnData = Encoding.ASCII.GetString(receiveBytes);

// Uses the IPEndPoint object to determine which of these two hosts responded.
Console.WriteLine("This is the message you received: \n" +
returnData.ToString());


udpClient.Close();
\end{verbatim}






%I clienten blev, der implementeret en construktor og recieveFile-metode.
%Constructoren tager to argumenter - et filnavn + stiangivelse og en ip-adresse til serveren.
%Det første, der sker i constructoren er, at der oprettes en ny TCPClient. Clienten forbinder derefter med en TCP-Server med den angivne IP-adresse og et hard-coded Portnummer.
%Clienten opretter derefter en networkstream, der skal tales med og sender en forespørgelse over at serveren skal sende den angivne fil til clienten. Clienten modtager derefter svar fra serveren om den angivne fil sendes på serveren. Hvis filens findes på serveren kalder clienten recieveFile og starter modtagelsen af filen.
%Efter filen er blevet overført lukker clienten forbindelsen til serveren.
%\begin{verbatim}
%private file_client (string[] args)
%{
%	long size;
%	Console.WriteLine ("Client started");
%	TcpClient clientSocket = new TcpClient ();
%	clientSocket.Connect (args[0], PORT);
%	NetworkStream serverStream = clientSocket.GetStream();
%	LIB.writeTextTCP (serverStream, args[1]); //Filename skal gives med som argument.
%	size = LIB.getFileSizeTCP(serverStream);
%	if (size != 0)
%		receiveFile (args [1], serverStream, size);
%	else
%		Console.WriteLine ("No such file exist on the server");
%
%	clientSocket.Close();
%}
%\end{verbatim}
%
%\noindent ReceiveFile (String fileName, NetworkStream io, long FileSize) 
%RecieveFile laver en fil med det pågældende filenavn, den har fået med som parameter. Den modtager også en networkstream, hvor dataene skal modtages fra, og så modtager den også størrelsen på filen som parameter.
%Metoden læser på networkstreamen og overfører dette til en buffer (Der bliver maksimalt aflæst 1000 bytes ad gangen). Når data er kommet ind i bufferen bliver de skrevet til den oprette fil. Denne proces fortsættes ind til, at der totalt er læst ligeså mange bytes som filesize.
%Man kan se koden for metoden nedenunder:
%
%\begin{verbatim}
%			private void receiveFile (String fileName, NetworkStream io, long fileSize)
%			{
%				var file = File.Create (fileName);
%				var buffer = new byte[BUFSIZE];
%				int bytesRead = 0;
%				long accumulatedBytes = 0;
%			
%				while (accumulatedBytes < fileSize) 
%				{
%					bytesRead = io.Read(buffer,0,BUFSIZE);
%					accumulatedBytes += bytesRead;
%			
%					file.Write (buffer, 0, bytesRead);
%				}
%			}
%\end{verbatim}